\documentclass[12pt]{article}


\usepackage[utf8]{inputenc}
\usepackage[T1]{fontenc}
\usepackage[francais]{babel}
\usepackage{hyperref}
\usepackage{pifont}


\title{Projet réseau - Chat Python}
\author{Loïc Lachiever}
\author{Romain Pigret-Cadou}

\begin{document}
\maketitle


\begin{abstract}
    \begin{center}
    \url{https://github.com/picachoc/IRC_python}
    \\Projet UE Réseau L2
    \\Université de Bordeaux
    \\Groupe TMA12\_13
    \\Année 2019
    \end{center}
\end{abstract}


\tableofcontents



\section{Introduction}
Dans le cadre de l'UE Réseau de la L2 informatique de l'Université de Bordeaux,
nous devions, pour nous initier aux sockets, réaliser un chat textuel en Python.
En s'inspirant fortement du protocole IRC, les fonctionnalités du chat nous étaient
imposées en 2 versions (0 et 1) via le cahier des charges disponible en Annexe.

\section{Fonctionnement général}

Deux scripts python sont donc nécessaires, le \textbf{client} et
le \textbf{serveur} (respectivement
client.py et server.py). Python a été préféré à d'autres langages pour sa simplicité
d'implémentation, il est vrai qu'il a été simple de mettre en place
certaines structures.

\underline{Le Serveur} a pour rôle d'orchestrer les différentes demandes des clients,
transmises par sockets. Il stocke aussi toutes les informations nécessaires comme
les canaux de discussion, les noms d'utilisateurs, qui est situé où, qui est admin, ...
Ces informations sont manipulées sous la forme de chaînes de caractères.
Pour communiquer avec le serveur il faut lui envoyer des commandes spécifiques,
de même il vous transmet ses informations via des commandes aussi (souvent le même intitulé).
En conséquence, pour se mettre d'accord sur ces commandes nous devions créer notre propre
protocole de communication \emph{(client <-> serveur)}.

\underline{Le Client} lui peut prendre diverse formes, que ce soit une application graphique
ou dans un terminal. L'intérêt du client est d'interpréter les commandes du protocole serveur
et de convertir en protocole serveur les ordres donnés par l'utilisateur, que se soit par un
clic sur un bouton ou en tapant "/commande". En effet le serveur est conçu pour fonctionner
parfaitement avec son protocole, d'ailleurs nous le testions avec la commande linux \emph{nc}.
Mais le client permet de rajouter une surcouche ergonomique, en traduisant les codes d'erreurs,
en parlant dans la langue utilisateur (français) et de manière plus claire que le protocole.

La conception du protocole serveur était totalement libre, seule la syntaxe et
la fonctionnalité des commandes
à taper dans le terminal client étaient imposées.


\section{Serveur}
Notre serveur est basé sur le protocole PICROM, intitulé par concaténation des
3 premières lettres de nos pseudonymes respectifs.

\subsection{Documentation}
Le serveur peut-être lancé depuis le shell IDLE ou sur un terminal linux avec
la commande : \emph{python3 server.py}
\\

\noindent{\textbf{\underline{Protocole Client -> Serveur :}}}
\\

\noindent{\underline{Légende:}}
\\\ding{88} : Nécessite les droits administrateurs
\\\ding{108} : Le client ne peut l'utiliser que dans un $channel$ (canal de discussion)
\\\ding{110} : Utilisable uniquement en dehors de tout $channel$, dans le $HUB$
\\\ding{115} : Utilisable de n'importe où

\begin{description}
    \item[---LIST]:\ding{115} le client demande la liste des $channels$ ouverts
    \item[---JOIN <channelName>]:\ding{115} le client veut rejoindre le canal "channelName" 
    \item[---LEAVE]:\ding{108} le client veut quitter le $channel$
    \item[---WHO] :\ding{108} le client veut la liste des $nicks$(pseudonymes) des clients
    sur le $channel$ courant
    \item[---MSG <message>]:\ding{108} le client veut envoyer un message sur le $channel$
    \item[---PRV\_MSG N <nick1> ... <nickN> <message>]:\ding{108} le client veut
         envoyer un message privé à N utilisateurs identifiés par leurs $nicks$
    \item[---BYE]:\ding{110} le client veut se déconnecter du serveur
    \item[---KICK <nick>] \ding{88}\ding{108} l'administrateur veut $kicker$ nick
    du $channel$: lui forcer le LEAVE
    \item[---REN <newChannelName>] \ding{88}\ding{108} l'administrateur veut renommer le $channel$
    \item[---CURRENT <newChannel |>] \ding{115} le client veut définir un autre
    $channel$ préalablement JOIN comme son $channel$ courant | ou simplement
    demander son $channel$ courant
    \item[---NICK <newNick>] \ding{115} le client veut changer son pseudo
    \item[---GRANT <newAdminNick>] \ding{88}\ding{108} l'administrateur veut rendre admin
                                                un client sur le $channel$ 
    \item[---REVOKE <oldAdminNick>] \ding{88}\ding{108} l'administrateur veut destituer 
                                            un admin sur le $channel$
    \\
\end{description}



\noindent{\textbf{\underline{Protocole Serveur -> Client :}}}
\\

\noindent{\underline{Légende:}}
\\<R> :  rang du client, 1 pour administrateur, 0 sinon
\\\ding{108} : Commande envoyée au client
\\\ding{110} : Commande envoyée aux autres clients du $channel$
\\\ding{115} : Commande envoyée à tout les autres clients du serveur
\\\ding{54} : Commande envoyée aussi au client"
\\\ding{88} : Commande envoyée au(x) client(s) cible(s)

\begin{description}
    \item[---ERR <code>] \ding{108} envoie d'un code d'erreur
    \item[---CONNECT <firstNick>] \ding{115}\ding{54} confirme la connexion
                                                        d'un nouveau client
    \item[---LIST <channel1> ... <channelN>] \ding{108} envoie la liste des $channels$
                                                ouverts, séparés par des espaces
    \item[---JOIN <channel> <R> <newCommerNick>] \ding{110}\ding{54} indique l'entrée
                                d'un nouveau client sur le $channel$, si votre rang est
                                1 alors vous venez d'ouvrir le $channel$
    \item[---LEAVE <channel> 0 <nick> | <channel> 1 <nick> <newAdmin>] \ding{110}\ding{54}
                indique qu'un client a quitté le $channel$ | indique un nouvel admin
                s'il était le dernier.
    \item[---WHO <R> <client1> ... <R> <clientN>] \ding{108} envoie la liste des clients
                                    sur le channel
    \item[---MSG <channel> <R> <nick> <message>] \ding{110} nick a envoyé un message sur le channel
    \item[---PRV\_MSG <channel> <R> <nick> <message>] \ding{88} nick vous a envoyé un message privé
    \item[---BYE <nick>] \ding{115} indique qu'un client s'est déconnecté du serveur
    \item[---KICK <channel> <adminNick> <R> <nick>] \ding{110}\ding{54} indique que nick va être
    immédiatement $kické$ par l'admin du $channel$
    \item[---REN <channel> <adminNick> <newName>] \ding{110}\ding{54} le $channel$ a été renommé par l'admin
    \item[---CURRENT <currentChannel>] \ding{108} indique au client son $channel$ courant ou celui qu'il vient de définir
    \item[---GRANT <channel> <adminNick> <newAdmin>] \ding{110}\ding{54} newAdmin a été promu admin par adminNick dans le $channel$
    \item[---REVOKE <channel> <adminNick> <oldAdmin>] \ding{110}\ding{54} oldAdmin n'est plus administrateur
    \\
\end{description}


\noindent{\textbf{\underline{Protocole erreurs :}}}

\begin{description}
    \item[ERR 0] Commande inconnue
    \item[ERR 1] Commande non autorisée, requiert les privilèges administrateurs
    \item[ERR 2] Vous essayez d'appliquer sur vous même une de ces commandes: KICK, PRV\_MSG, GRANT ou REVOKE
    \item[ERR 3] Ce pseudo est déjà utilisé
    \item[ERR 4] Le client sélectionné n'est pas sur le $channel$
    \item[ERR 5] Commande non autorisée ici, rejoignez un $channel$ ou quittez les tous. 
    \item[ERR 6] Vous n'avez pas rejoins ce $channel$, impossible d'en faire le $channel$ courant
    \item[ERR 7] Vous devez d'abord choisir un pseudo avant toute opération sur le serveur
    \item[ERR 8] Ce nom de $channel$ est déjà pris
    \item[ERR 9] Arguments mal saisis
    \item[ERR 10] Vous essayez de JOIN ou CURRENT le $channel$ spécial "HUB"
    \item[ERR 11] Vous essayez de JOIN un $channel$ déjà rejoint, ou de CURRENT votre $channel$ courant
    \item[ERR 12] Vous voulez GRANT un admin ou REVOKE un non-admin
\end{description}


\subsection{Implémentation}


\section{Client}

\subsection{Documentation}

\subsection{Implémentation}



\section{Conclusion}




\section{Annexes}
\subsection{Cahier des charges}
{\large \textbf{\underline{Commandes client v0:}}}
\begin{itemize}
    \item /HELP: print this message
    \item /LIST: list all available channels on server
    \item /JOIN <channel>: join (or create) a channel
    \item /LEAVE: leave current channel
    \item /WHO: list users in current channel
    \item <message>: send a message in current channel
    \item /MSG <nick> <message>: send a private message in current channel
    \item /BYE: disconnect from server
    \item /KICK <nick>: kick user from current channel [admin]
    \item /REN <channel>: change the current channel name [admin]
\end{itemize}

{\large \textbf{\underline{Commandes client v1:}}}
\begin{itemize}
    \item /CURRENT: print current channel name
\item /CURRENT <channel>: set current channel
\item /MSG <nick1;nick2;...> <message>: send a private message to several users in current channel
\item /NICK <nick>: change user nickname on server
\item /GRANT <nick>: grant admin privileges to a user [admin]
\item /REVOKE <nick>: revoke admin privileges [admin]
\item /SEND <nick> </path/to/file>: send a file to a remote user
\item /RECV </path/to/file>: receive a file and save it locally
\item /HISTORY: print history of current channel (saved by server)
\end{itemize}


\end{document}

